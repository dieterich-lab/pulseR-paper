First-order reaction kinetics  is one of simplified models,
which can help to describe RNA dynamics \emph{in vivo}\cite{A}.
Given
\begin{itemize}
 \item constant synthesis rate $s$ and
\item degradation rate $d$, 
\end{itemize}
RNA concentration $r$ follows the ordinary differential equation
\begin{equation}
 \dot{r} = s - dr,
\end{equation}
where $\dot{r}$ stands for the time derivative of the $r$\cite{A}.
\par
During synthesis, a new RNA molecule incorporates labelled uridine bases\cite{A}.
For zero initial condition $r_L(0) = 0$, the solution is
\begin{equation}
 r_\text{L}(t) = \frac{s}{d}\left(1 - e^{-dt}\right).
\end{equation}
With time, the labelled fraction tends to the steady state level of concentration $\mu$,
$\lim_{t\to\infty} r_\text{L}(t) = \frac{s}{d} = \mu$.
In contrast, the unlabelled molecules are only being degraded during the \emph{pulse}-experiment.
Hence, assuming initial level of unlabelled RNA to be the steady-state one, $r_U = \mu$,
the the amount of unlabelled fraction at a time $t$ is
\begin{equation}
 r_\text{U}(t) = \mu e^{-dt}.
\end{equation}
The example model includes only two parameters and does not consider
RNA maturation and existence of several isoforms. For more complex approaches we refer to \cite{A}.
\par For completeness we provide the formulas, which describe expression levels for \emph{chase}-experiments.
In this case, we assume that no synthesis of labelled RNA occurs after labelling period $t_L$:
\begin{align}
 r_\text{T}&=\mu\\
 r_\text{L}&=\mu \left(1-e^{-dt_\text{L}}\right)e^{-dt_\text{C}}\\
 r_\text{L}&=\mu \left(1-\left(1-e^{-dt_\text{L}}\right)e^{-dt_\text{C}}\right),
\end{align}
where $t_\text{C}$ stands for the longitude of the chase period.
\par

%TODO
% - NB
% - fraction factors
% - contamination
% 
