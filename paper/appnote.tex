RNA dynamics can be described by ordinary differential equations,
which have simple analytic solution 
in the simplest cases such as degradation or synthesis with 
the constant rate. In the \verb|pulseR| package,
the user need to specify the expressions for the expected RNA abundances,
Alternatively, formulas can be generated using package functions for the 
simple cases. For example, if the steady-state RNA level is $\mu$, then
during pulse-experiment the labelled fraction will evolve as
\begin{equation}
 [\text{Labelled}] \sim \mu (1 - e^{-dt}),
\end{equation}
where $d$ is the degradation rate.
\par The aim of the package is to simplify the procedure of the parameter estimating.
Usually, the most interest is focused on the gene-specific parameters ($d$ and $\mu$
in the example above). However, some parameters can be shared between all genes and
our package is able to treat them accordingly. This can be useful, if 
one would like to take into account difference in the uridine content, since
it can introduce a bias in the estimations, \citep{miller2011dynamic,
schwalb2012measurement}. In this case, the RNA abundances are multiplied by 
a factor which depends on the number of uridine bases in the transcript $\#u$,
$(1 − (1 − p)^{\#u})$, where $p$ is the probability of an uridine to be replcaed
by 4sU.

