\documentclass{bioinfo}
\copyrightyear{2015} \pubyear{2015}

\access{Advance Access Publication Date: Day Month Year}
\appnotes{category}


\begin{document}
\firstpage{1}

\subtitle{Application Note}

\title[pulseR package]{pulseR: Versatile computational analysis of RNA turnover from metabolic labeling experiments}
\author[Uvarovskii \textit{et~al}.]{
Uvarovskii Alexey\,$^{\text{\sfb 1,2}*}$,
Christoph Dieterich\,$^{\text{\sfb 1,2}*}$ }
\address{$^{\text{\sf 1}}$
Section of Bioinformatics and Systems Cardiology,
Klaus Tschira Institute for Integrative Computational Cardiology,
Department of Internal Medicine III,
University Hospital Heidelberg,   
Im Neuenheimer Feld 669,
69120 Heidelberg,
and 
$^{\text{\sf 2}}$
German Center for Cardiovascular Research (DZHK),
Im Neuenheimer Feld 669
69120 Heidelberg
}

\corresp{$^\ast$To whom correspondence should be addressed.}

\history{Received on XXXXX; revised on XXXXX; accepted on XXXXX}

\editor{Associate Editor: XXXXXXX}

\abstract{\textbf{Motivation:} 
Metabolic labelling of RNA is a well established and powerful method to estimate RNA synthesis and decay rates.
The pulseR R package simplifies the analysis of RNA-seq count data that emerges from corresponding pulse-chase experiments. \\
\textbf{Results:} 
The pulseR package provides a flexible interface and readily accommodates numerous different experimental designs.
To our knowledge, it is the first publicly available software solution that models count data with the more appropriate negative-binomial model.
Moreover, pulseR handles both, labelled and unlabelled, spike-ins in its workflow and accounts for potential labeling biases (e.g. number of uridine residues). \\
\textbf{Availability:} The pulseR package is freely
available at https://github.com/dieterich-lab/pulseR  under the
GPLv3.0 licence \\
\textbf{Contact:} \href{a.uvarovskii@uni-heidelberg.de}{a.uvarovskii@uni-heidelberg.de} and \href{christoph.dieterich@uni-heidelberg.de}{christoph.dieterich@uni-heidelberg.de}\\
\textbf{Supplementary information:} Supplementary data are available \href{https://github.com/dieterich-lab/pulseR-paper/blob/master/R/supplementary.pdf}{https://goo.gl/xmonsg}}
\maketitle

\section*{Introduction}
Gene expression is a dynamic process. Experiments which aim to measure
kinetics of expression levels in time can help to reveal mechanisms of gene regulation. 
With  new experimental set-ups, new types of generated data occur and one needs
to develop new methods and frameworks to be able to extract the knowledge out of 
the results.
\par 
Pulse-chase experimental approach allow to track changes 
of mRNA due to synthesis of nascent RNA molecules and
degradation of the old ones. The idea of the method is to introduce a label, which 
incorporates in the newly synthesised RNA molecules and can be traced later.
4sU labelling introduced by \citep{dolken2008high} allowed to estimate 
kinetic rates of RNA metabolism in a number of studies up to date 
\citep{
sabo2014selective,
rabani2011metabolic,
miller2011dynamic,
schwanhausser2011global,
eser2016determinants,
schwalb2016tt,
marzi2016degradation,
zhang2016biogenesis,
neymotin2014determination,
mukherjee2016integrative}.

reviewed in \citep{wachutka2016measures}
%\vspace*{-.5cm}
\section*{Implementation}
RNA dynamics can be described by ordinary differential equations,
which have simple analytic solution 
in the simplest cases such as degradation or synthesis with 
the constant rate. In the \verb|pulseR| package,
the user need to specify the expressions for the expected RNA abundances,
Alternatively, formulas can be generated using package functions for the 
simple cases. For example, if the steady-state RNA level is $\mu$, then
during pulse-experiment the labelled fraction will evolve as
\begin{equation}
 [\text{Labelled}] \sim \mu (1 - e^{-dt}),
\end{equation}
where $d$ is the degradation rate.
\par The aim of the package is to simplify the procedure of the parameter estimating.
Usually, the most interest is focused on the gene-specific parameters ($d$ and $\mu$
in the example above). However, some parameters can be shared between all genes and
our package is able to treat them accordingly. This can be useful, if 
one would like to take into account difference in the uridine content, since
it can introduce a bias in the estimations, \citep{miller2011dynamic,
schwalb2012measurement}. In this case, the RNA abundances are multiplied by 
a factor which depends on the number of uridine bases in the transcript $\#u$,
$(1 − (1 − p)^{\#u})$, where $p$ is the probability of an uridine to be replcaed
by 4sU.


\vspace*{-.5cm}
\section*{Discussion}
\subsection{Comparison to existing approaches}
The published so far approaches are different in terms of 
data normalisation, statistical model and underlying mathematical model of the 
RNA metabolism.
\subsubsection*{Normalisation}
The normalisation step aims to uncover relations between samples, which can 
be different in sequencing depth, represent different fractions and were subjected
different protocols of preparation. 
Spike-in molecules, which one adds before fraction separation step,
allow to simplify the fitting procedure. In this case, estimation of fraction-specific 
coefficients and cross-contamination rates is separated from fitting of gene-specific parameters
\citep{schwalb2016tt,neymotin2014determination}.
\par 
However, it comes with the price of complication of the exprerimental procedure,
because one needs to synthesis labelled spike-ins molecules \citep{}.
Alternative approach is to determine fraction- or sample- specific 
normalising coefficients on the basis of the gene expression data only.
 In most cases, the analysis is performed on transformed raw read counts, i.e.
 to RPKMs (Reads Per Kilobase of transcript per Million).
\begin{itemize}
 \item 
\textbf{normalisation to RNA quantities:}  the total amount of 
RNA  is measured for every fraction. This ratios of RNA quantity between 
different fractions are used as fraction-specific coefficients \citep{rabani2011metabolic}.
\item 
\textbf{normalisation via regression:} \citet{schwanhausser2011global} 
used an approach which was previously applied to 
 the microarray data in \citep{dolken2008high}.
 The total fraction $T$ is related to the labelled $L$ and unlabelled $U$ ones as
 $T = aU + bL$, where $a$ and $b$ are unknown.
 Ordinary regression recovers the coefficients $a$ and $b$. However,
 such procedure can results in unrealistic fraction ratios, and \citet{schwanhausser2011global}
 had to exclude certain genes from the analysis. 
 The total least-squares fitting is an another alternative \citep{miller2011dynamic, schwalb2012measurement}.
 \item \textbf{normalisation coupled with model fitting:}
 coefficients can be introduced as model parameters. Their values are
 recovered by MLE  together with other parameters \citep{eser2016determinants, de2015inspect}.
 Although being harder to implement, this approach allows parameters to influence 
 on each other, and hence, can result in  better estimations.
\end{itemize}
Our package can be used either with spike-in or spike-in free data.
Additional known or unknown shared parameters can be introduced by a user as well, 
which allows high flexibility in the experiment analysis.

\subsubsection*{Statistical model}
\textbf{The NB distribution} is known to describe well the count data in RNA-seq 
experiments \citep{robinson2007moderated}. This encourages to use it over 
normal distribution assumption, which was applied earlier to the microarray data sets
\citep{miller2011dynamic}. In our package, we implement NB assumptions, which were
successfully applied to short pulse-labelling in \citep{eser2016determinants,
schwalb2016tt}.\\
\textbf{The normal distribution} is implemented in \verb|INSPEcT| R package \citep{de2015inspect},
\verb|DRiLL| software \citep{rabani2014high}. 
Although such assumptions 
were applied to RNA-seq data previously \citep{}, we think that this approach suits better 
for microarray data analysis.\\
\verb|DTA| R package \citep{schwalb2012measurement} is designed to
estimate kinetic rates from a single time point. The values for 
total, labelled and unlabelled fractions are treated as known parameters, and 
no statistical model is assumed. A similar approach is implemented in 
\verb|HALO| Java framework \citep{friedel2010halo}. 

\subsubsection*{Mathematical model}
In our package implementation, we leave freedom for user to formulate a mathematical
model for the data. In the \emph{methods} section we describe a model, which 
does not include RNA maturation.


- no delays in the model (transcr-transl)					



\begin{table}
 \begin{tabular}{|c|c|c|c|c|c|}\hline
                        &pulseR &DRiLL          &INSPEcT&DTA    &HALO       \\\hline
 stat.model             & NB    &normal,bin     &normal  & normal&normal     \\\hline                         
 spike-ins              & yes   & no            &  no    &  no   & no        \\\hline               
 several time points    & yes   & yes           &  yes   &  no   & no        \\\hline                    
 ODE integration        & no$^1$& yes           &  yes   &  no   & no        \\\hline 
 experiment flexibility & yes   & no            &  no    &  no   & no        \\\hline 
 variable rates in time & no    & yes           &  yes   &  no   & no        \\\hline 
            uridine bias& yes   & no            &  no    &  yes  & yes       \\\hline 
       RNA processing   & no$^2$& yes           &  yes   &  no   & no        \\\hline 
  language              & R     &MATLAB         &  R     &  R    & Java      \\\hline 
 \end{tabular}

\end{table}

\subsection*{Acknowledgements}
AU and CD would like to thank all members from the Dieterich Lab for their great input.
Special thanks go to Isabel Naarman-de Vries for all experimental support and insights.
\paragraph{Funding\textcolon} 
The work of AU and CM was kindly supported 
by the Klaus Tschira Stiftung gGmbH and the German Center for Cardiovascular Research (DZHK).
\vspace*{-0.6cm}
\bibliographystyle{natbib}
%\bibliographystyle{achemnat}
%\bibliographystyle{plainnat}
%\bibliographystyle{abbrvnat}
%\bibliographystyle{bioinformatics}
%
%\bibliographystyle{plain}
%
\bibliography{ref.bib}


\end{document}
