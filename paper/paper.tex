\documentclass{bioinfo}
\copyrightyear{2015} \pubyear{2015}

\access{Advance Access Publication Date: Day Month Year}
\appnotes{category}


\begin{document}
\firstpage{1}

\subtitle{Application Note}

\title[pulseR package]{pulseR: Versatile computational analysis of RNA turnover from metabolic labeling experiments}
\author[Uvarovskii \textit{et~al}.]{
Uvarovskii Alexey\,$^{\text{\sfb 1,2}*}$,
Christoph Dieterich\,$^{\text{\sfb 1,2}*}$ }
\address{$^{\text{\sf 1}}$
Section of Bioinformatics and Systems Cardiology,
Klaus Tschira Institute for Integrative Computational Cardiology,
Department of Internal Medicine III,
University Hospital Heidelberg,   
Im Neuenheimer Feld 669,
69120 Heidelberg,
and 
$^{\text{\sf 2}}$
German Center for Cardiovascular Research (DZHK),
Im Neuenheimer Feld 669
69120 Heidelberg
}

\corresp{$^\ast$To whom correspondence should be addressed.}

\history{Received on XXXXX; revised on XXXXX; accepted on XXXXX}

\editor{Associate Editor: XXXXXXX}

\abstract{\textbf{Motivation:} 
Metabolic labelling of RNA is a well established and powerful method to estimate RNA synthesis and decay rates.
The pulseR R package simplifies the analysis of RNA-seq count data that emerges from corresponding pulse-chase experiments. \\
\textbf{Results:} 
The pulseR package provides a flexible interface and readily accommodates numerous different experimental designs.
To our knowledge, it is the first publicly available software solution that models count data with the more appropriate negative-binomial model.
Moreover, pulseR handles both, labelled and unlabelled, spike-ins in its workflow and accounts for potential labeling biases (e.g. number of uridine residues). \\
\textbf{Availability:} The pulseR package is freely
available at https://github.com/dieterich-lab/pulseR  under the
GPLv3.0 licence \\
\textbf{Contact:} \href{a.uvarovskii@uni-heidelberg.de}{a.uvarovskii@uni-heidelberg.de} and \href{christoph.dieterich@uni-heidelberg.de}{christoph.dieterich@uni-heidelberg.de}\\
\textbf{Supplementary information:} Supplementary data are available \href{https://github.com/dieterich-lab/pulseR-paper/blob/master/R/supplementary.pdf}{https://goo.gl/xmonsg}}
\maketitle

\section*{Introduction}
Gene expression level is defined by the rates of RNA synthesis and degradation.
Understanding how certain gene levels are regulated in a response to
 condition changes might help to uncover the underlying control mechanisms.
\par
The pulse-chase experimental approach allows to measure such kinetics. 
 In this method, tracing molecules are introduced to the medium, 
 which results in their incorporation into nascent RNA molecules.
4sU labelling, developed by \citep{dolken2008high}, is used to estimate 
kinetic rates of RNA metabolism in a number of studies up to date, 
see \citep{wachutka2016measures} for the review.
The  RNA-seq data generated in such experiments  have a discrete nature.
However, there is no software available for  parameter
estimation in kinetic models of gene expression, which is specifically designed to handle 
count data. Here we present the \verb|pulseR| package, which allows to 
process RNA-seq data from 4sU-labelling experiments.
%\citep{
%sabo2014selective,
%rabani2011metabolic,
%miller2011dynamic,
%schwanhausser2011global,
%eser2016determinants,
%schwalb2016tt,
%marzi2016degradation,
%zhang2016biogenesis,
%neymotin2014determination,
%mukherjee2016integrative}.
%\vspace*{-.5cm}
\section*{Implementation}
\subsubsection*{Parameter definition}
RNA dynamics can be described by ordinary differential equations,
which have simple analytic solution 
if the degradation and synthesis rates are assumed to be constant.
In the \verb|pulseR| package,
users need to specify the expressions for the mean RNA abundances,
Alternatively, formulas can be generated using package functions for the 
most frequent cases (\textcolor{red}{how?}). 

Although the most interest is focused on the gene-specific parameters,
pulseR allows to introduce shared parameters.
 This can be useful for taking into account the difference in the uridine content, since
it can introduce a bias in the estimations \citep{miller2011dynamic,
schwalb2012measurement}.
In this case, the RNA abundances are multiplied by 
a probability that at least one uridine in the molecule is substituted by 4sU.
The shared parameter then is the probability for a single base to be substituted
by a 4sU.
%\begin{equation}
%P(\text{\#u}, p) = 1-(1-p)^{\text{\#}u},
%\end{equation}
%where $p$ is the probability of a single replacement and  
% $\text{\#}u$ is the number of uridine bases in the molecule.
\subsubsection*{Normalisation}
We introduce additional parameters (\textcolor{red}{Which ?}) to account for different sequencing depths.
Additionally, the pull-down procedure will have an effect on the amount and purity of captured RNA \textcolor{red}{Which parameter - please add a list of parameters to supplement?}. 
 For example, 
if the labelled fraction consists of the labelled RNA $L_{ij}$ and the unlabelled RNA
 $U_{ij}$ molecules, for a sample $j$ and gene $i$ we have 
\begin{equation}
 [\text{labelled fraction}]_{ij}  = \alpha_{j} L_{ij} + \beta_{j} U_{ij}
\end{equation}
In case spike-ins are present, $\alpha_{j}$ and and $\beta_j$ can be directly estimated from spike-in read counts. 
To this end, the user provides lists of spike-ins which represent unlabeled or labeled RNA.
\par
In the absence of spike-ins, 
normalization factors are derived from gene counts
because the system is overdetermined.
Inside a given RNA-seq group (e.g. [total, labeled, unlabeled ] x conditions) samples are normalized for sequencing depth $d_j$ following the DESeq procedure.
Normalization between the groups is performed during the fitting procedure, 
and these coefficients $\alpha$ and $\beta$ are shared between the samples from the same group:
\begin{equation}
 [\text{labelled fraction}]_{ij}  = d_j(\alpha L_{ij} + \beta U_{ij})
\end{equation}

\subsubsection*{Parameter estimation}
We use the maximum likelihood method (MLE) to obtain parameter values.
A typical RNA-seq experiment estimates gene abundance levels by read counts.
It has been previously shown that read counts are well represented by a negative-binomial model,
which takes over-dispersion into account 
\citep{robinson2007moderated}.
The NB distribution has two parameters, the mean $m$ and the dispersion parameter 
$\alpha$.
Hence,  a read number of a gene $i$ in a sample $j$ follows
\begin{equation}
 K_{ij} \sim \text{NB}(m_{ij}, \alpha).
\end{equation}
\textcolor{red}{alpha above and alpha here mean different things - what about sigma}
The dispersion parameters $\alpha$ is shared between all
samples and genes. Otherwise it would not be possible to infer all parameters 
from a small number of replicates (usually, only 2 or 3 points are available).

We separated the fitting procedure into several simpler steps:
\begin{enumerate}
 \item fitting of gene-specific parameters (e.g. degradation rate)
 \item fitting of shared parameters  
 \item fitting of the normalization factors (for a spike-in-free design)
 \item estimation of the dispersion parameter 
\end{enumerate}
We repeat the steps 1-4 until user-specified convergence criteria are met.
We do not consider gene-gene interactions in this model, but it is possible to 
 fit this parameters independently in future work. 

We optimise the likelihood functions by using the {L-BFGS-U} method \citep{byrd1995limited}, which is
available in the \verb|stats| R package \citep{rlang}.
\vspace*{-.5cm}
\section*{Discussion}
\subsubsection*{Comparison with existing approaches}
All published approaches differ in terms of 
data normalization, statistical model and level of detail with regards to 
RNA metabolism. We have compared the following software packages with \verb|pulseR| :
DRiLL \citep{rabani2014high},
INSPEcT \citep{de2015inspect},
DTA \citep{schwalb2012measurement},
HALO \citep{friedel2010halo}.
In most cases, samples are normalized by utilizing overdetermination of the system.
The normalization coefficients are either estimated via regression
(DTA, HALO) or during the MLE procedure together  
with other parameters (INSPEcT, DRiLL). 
Additionally, 
\verb|pulseR| allows to use spike-ins counts as an alternative normalization strategy.
HALO and DTA estimate degradation rates simply as  a ratio of labelled and
total RNA fractions.
In DRiLL, expression levels are fitted to the binomial distribution. 
However, the kinetic rates are estimated via optimization of residual sum of squares in both, DRiLL and 
INSPEcT.
In contrast, \verb|pulseR| assumes the NB distribution
in MLE of all parameters, which allows to work directly with count data.
Experiments may vary in design and the number of conditions and \verb|pulseR| offers unprecedented flexibility.
While DTA and HALO estimate rates on the basis of a single time point only (using all three fractions),
\verb|pulseR|, DRiLL and INSPEcT can infer rates from several time points.
While pulseR can handle different designs including pulse, chase- or combined 
experiments, all other packages work only with pulse experiments.
However, the DRiLL and INSPEcT packages can model 
time-dependent rates out of the box.
Moreover, the DRiLL software is able to model the gene-transcript dependence structure.
Table 1 summarizes all relevant software features.
\begin{table}
 \begin{tabular}{|l|c|c|c|c|c|}\hline
                        &pulseR &DRiLL          &INSPEcT&DTA    &HALO       \\\hline
 statistical model      & NB    &N, BIN         &N       & -    & -      \\\hline                         
 spike-ins              & +     &   -           &  -     &  -    & -         \\\hline               
 several time points    & +     &   +           &  +     &  -    & -         \\\hline                    
  variable design $^1$       & +     &   -           &  -     &  -    & -         \\\hline 
 non-constant rates     & -     &   +           &  +     &  -    & -         \\\hline 
            uridine bias& +     &   -           &  -     &  +    & +         \\\hline 
       RNA processing   &$\ast$ &   +           &  +     &  -    & -         \\\hline 
       gene isoforms   & $\dagger$     &   +           &  -     &  -    & -         \\\hline 
  language              & R     &MATLAB         &  R     &  R    & Java      \\\hline 
 \end{tabular}
\caption{Comparison of available software for parameter estimation in 
pulse-chase experiments. N: normal, NB: negative binomial, BIN: binomial.
$\ast$ - must be defined by a user. $^1$ - pulse, chase or combination thereof experiments.
$\dagger$ - count estimates on isoform level can be used (preprocessing).
}
\end{table}
We have evaluated \verb|pulseR| performance using simulated data \textcolor{red}{Source ?}. 
For a detailed description of the workflow and results please refer to the supplementary material.

%\section{Conclusion}
% The pulseR package provide an  interface for fitting models of
% RNA dynamics to the read count data. 
% It is freely available at 
% http://github.com/dieterich-lab/pulseR.
 \vspace*{-.3cm}
\subsection*{Acknowledgements}
AU and CD would like to thank all members from the Dieterich Lab for their great input.
Special thanks go to Isabel Naarman-de Vries for all experimental support and insights.
\paragraph{Funding\textcolon} 
The work of AU and CM was kindly supported 
by the Klaus Tschira Stiftung gGmbH and the German Center for Cardiovascular Research (DZHK).
\vspace*{-0.6cm}
\bibliographystyle{natbib}
%\bibliographystyle{achemnat}
%\bibliographystyle{plainnat}
%\bibliographystyle{abbrvnat}
%\bibliographystyle{bioinformatics}
%
%\bibliographystyle{plain}
%
\bibliography{ref.bib}


\end{document}
