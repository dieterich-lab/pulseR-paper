\subsection{Comparison to existing approaches}
The published so far approaches are different in terms of 
data normalisation, statistical model and underlying mathematical model of the 
RNA metabolism.
\subsubsection*{Normalisation}
The normalisation step aims to uncover relations between samples, which can 
be different in sequencing depth, represent different fractions and were subjected
different protocols of preparation. 
Spike-in molecules, which one adds before fraction separation step,
allow to simplify the fitting procedure. In this case, estimation of fraction-specific 
coefficients and cross-contamination rates is separated from fitting of gene-specific parameters
\citep{schwalb2016tt,neymotin2014determination}.
\par 
However, it comes with the price of complication of the exprerimental procedure,
because one needs to synthesis labelled spike-ins molecules \citep{}.
Alternative approach is to determine fraction- or sample- specific 
normalising coefficients on the basis of the gene expression data only.
 In most cases, the analysis is performed on transformed raw read counts, i.e.
 to RPKMs (Reads Per Kilobase of transcript per Million).
\begin{itemize}
 \item 
\textbf{normalisation to RNA quantities:}  the total amount of 
RNA  is measured for every fraction. This ratios of RNA quantity between 
different fractions are used as fraction-specific coefficients \citep{rabani2011metabolic}.
\item 
\textbf{normalisation via regression:} \citet{schwanhausser2011global} 
used an approach which was previously applied to 
 the microarray data in \citep{dolken2008high}.
 The total fraction $T$ is related to the labelled $L$ and unlabelled $U$ ones as
 $T = aU + bL$, where $a$ and $b$ are unknown.
 Ordinary \citep{schwanhausser2011global} or recovers
 the coefficients $a$ and $b$. 
 \item CHANGE ME total least-squares \citep{miller2011dynamic}
 \item 
\end{itemize}




- no delays in the model (transcr-transl)					
