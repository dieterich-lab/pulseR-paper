\subsection{Comparison to existing approaches}
The published approaches are different in terms of 
data normalisation, statistical model and underlying mathematical model of the 
RNA metabolism. Here we analyse the following software:
DRiLL \citep{rabani2014high},
INSPEcT \citep{de2015inspect},
DTA \citep{schwalb2012measurement},
HALO \citep{friedel2010halo}.
\subsubsection*{Normalisation}
Samples are mostly normalised utilising overdetermination of the system.
The normalisation coefficients are estimated via regression
(DTA, HALO?) or during the fitting together 
with other parameters (INSPEcT, DRiLL). 
Besides normalisation coupled with the parameter fitting, 
the pulseR allows to use spike-ins counts as an alternative.

\subsubsection*{Statistical model}

Since such models were first applied to microarray pulse-experiments,
the normal distribution assumptions was used (DTA, HALO).
In DRiLL, expression levels are fitted to the binomial distribution. 
However, the kinetic rates are estimated via
optimisation of residual sum of squares in both, DRiLL and 
INSPEcT.
This implies the normality of the noise.
In contrast, the pulseR implements the negative binomial distribution
for fitting all the parameters,
which allows to work directly on the count data.
\subsubsection*{Other features}
\begin{table}
 \begin{tabular}{|c|c|c|c|c|c|}\hline
                        &pulseR &DRiLL          &INSPEcT&DTA    &HALO       \\\hline
 stat.model             & NB    &N, BIN         &N       & N     &N      \\\hline                         
 spike-ins              & +     &   -           &  -     &  -    & -         \\\hline               
 several time points    & +     &    +          &  +     &  -    & -         \\\hline                    
 experiment flexibility & +     &   -           &  -     &  -    & -         \\\hline 
 variable rates in time &  -    &    +          &  +     &  -    & -         \\\hline 
            uridine bias& +     &   -           &  -     &  +    & +         \\\hline 
       RNA processing   & -     &    +          &  +     &  -    & -         \\\hline 
  %language              & R     &MATLAB         &  R     &  R    & Java      \\\hline 
 \end{tabular}
\caption{Comparison of available software for parameter estimation in 
pulse-chase experiments. N: normal, NB: negative binomial, BIN: binomial.}
\end{table}

\subsection{Application}
We evaluated \verb|pulseR| performance using simulated data. 
The software is able to reproduce gene-specific parameters and 
sample normalisation factors.
For the detailed
description of the workflow please refer to the supplementary material.

\section{Conclusion}