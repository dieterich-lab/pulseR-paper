\subsection{Comparison to existing approaches}
The published so far approaches are different in terms of 
data normalisation, statistical model and underlying mathematical model of the 
RNA metabolism.
\subsubsection*{Normalisation}
The normalisation step aims to uncover relations between samples, which can 
be different in sequencing depth, represent different fractions and were subjected
different protocols of preparation. 
Spike-in molecules, which one adds before fraction separation step,
allow to simplify the fitting procedure. In this case, estimation of fraction-specific 
coefficients and cross-contamination rates is separated from fitting of gene-specific parameters
\citep{schwalb2016tt,neymotin2014determination}.
\par 
However, it comes with the price of complication of the exprerimental procedure,
because one needs to synthesis labelled spike-ins molecules \citep{}.
Alternative approach is to determine fraction- or sample- specific 
normalising coefficients on the basis of the gene expression data only.
 In most cases, the analysis is performed on transformed raw read counts, i.e.
 to RPKMs (Reads Per Kilobase of transcript per Million).
\begin{itemize}
 \item 
\textbf{normalisation to RNA quantities:}  the total amount of 
RNA  is measured for every fraction. This ratios of RNA quantity between 
different fractions are used as fraction-specific coefficients \citep{rabani2011metabolic}.
\item 
\textbf{normalisation via regression:} \citet{schwanhausser2011global} 
used an approach which was previously applied to 
 the microarray data in \citep{dolken2008high}.
 The total fraction $T$ is related to the labelled $L$ and unlabelled $U$ ones as
 $T = aU + bL$, where $a$ and $b$ are unknown.
 Ordinary regression recovers the coefficients $a$ and $b$. However,
 such procedure can results in unrealistic fraction ratios, and \citet{schwanhausser2011global}
 had to exclude certain genes from the analysis. 
 The total least-squares fitting is an another alternative \citep{miller2011dynamic, schwalb2012measurement}.
 \item \textbf{normalisation coupled with model fitting:}
 coefficients can be introduced as model parameters. Their values are
 recovered by MLE  together with other parameters \citep{eser2016determinants, de2015inspect}.
 Although being harder to implement, this approach allows parameters to influence 
 on each other, and hence, can result in  better estimations.
\end{itemize}
Our package can be used either with spike-in or spike-in free data.
Additional known or unknown shared parameters can be introduced by a user as well, 
which allows high flexibility in the experiment analysis.

\subsubsection*{Statistical model}
\textbf{The NB distribution} is known to describe well the count data in RNA-seq 
experiments \citep{robinson2007moderated}. This encourages to use it over 
normal distribution assumption, which was applied earlier to the microarray data sets
\citep{miller2011dynamic}. In our package, we implement NB assumptions, which were
successfully applied to short pulse-labelling in \citep{eser2016determinants,
schwalb2016tt}.\\
\textbf{The normal distribution} is implemented in \verb|INSPEcT| R package \citep{de2015inspect},
\verb|DRiLL| software \citep{rabani2014high}. 
Although such assumptions 
were applied to RNA-seq data previously \citep{}, we think that this approach suits better 
for microarray data analysis.\\
\verb|DTA| R package \citep{schwalb2012measurement} is designed to
estimate kinetic rates from a single time point. The values for 
total, labelled and unlabelled fractions are treated as known parameters, and 
no statistical model is assumed. A similar approach is implemented in 
\verb|HALO| Java framework \citep{friedel2010halo}. 

\subsubsection*{Mathematical model}
In our package implementation, we leave freedom for user to formulate a mathematical
model for the data. In the \emph{methods} section we describe a model, which 
does not include RNA maturation.


- no delays in the model (transcr-transl)					



\begin{table}
 \begin{tabular}{|c|c|c|c|c|c|}\hline
                        &pulseR &DRiLL  &INSPEcT&DTA    &HALO       \\\hline
 stat.model             & NB    &normal&normal  & normal&normal     \\\hline                         
 spike-ins              & yes   & no   &  no    &  no   & no        \\\hline               
 several time points    & yes   & yes  &  yes   &  no   & no        \\\hline                    
 ODE integration        & no$^1$& yes  &  yes   &  no   & no        \\\hline 
 experiment flexibility & yes   & no   &  no    &  no   & no        \\\hline 
 variable rates in time & no    & yes  &  yes   &  no   & no        \\\hline 
 can handle uridine bias& yes   & no   &  no    &  yes  & no        \\\hline 
 model RNA processing   & no$^2$& yes  &  yes   &  no   & no        \\\hline 
 \end{tabular}

\end{table}
