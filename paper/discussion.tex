\subsubsection*{Comparison with existing approaches}
All published approaches differ in terms of 
data normalization, statistical model and level of detail with regards to 
RNA metabolism. We have compared the following software packages with \verb|pulseR| :
DRiLL \citep{rabani2014high},
INSPEcT \citep{de2015inspect},
DTA \citep{schwalb2012measurement},
HALO \citep{friedel2010halo}.
In most cases, samples are normalized by utilizing overdetermination of the system.
The normalization coefficients are either estimated via regression
(DTA, HALO) or during the MLE procedure together  
with other parameters (INSPEcT, DRiLL). 
Additionally, 
\verb|pulseR| allows to use spike-ins counts as an alternative normalization strategy.
HALO and DTA estimate degradation rates simply as  a ratio of labelled and
total RNA fractions.
In DRiLL, expression levels are fitted to the binomial distribution. 
However, the kinetic rates are estimated via optimization of residual sum of squares in both, DRiLL and 
INSPEcT.
In contrast, \verb|pulseR| assumes the NB distribution
in MLE of all parameters, which allows to work directly with count data.
Experiments may vary in design and the number of conditions and \verb|pulseR| offers unprecedented flexibility.
While DTA and HALO estimate rates on the basis of a single time point only (using all three fractions),
\verb|pulseR|, DRiLL and INSPEcT can infer rates from several time points.
While pulseR can handle different designs including pulse, chase- or combined 
experiments, all other packages work only with pulse experiments.
However, the DRiLL and INSPEcT packages can model 
time-dependent rates out of the box.
Moreover, the DRiLL software is able to model the gene-transcript dependence structure.
Table 1 summarizes all relevant software features.
\begin{table}[tbh]
 \begin{tabular}{|l|c|c|c|c|c|}\hline
                        &pulseR &DRiLL          &INSPEcT&DTA    &HALO       \\\hline
 statistical model      & NB    &N, BIN         &N       & -    & -      \\\hline                         
 spike-ins              & +     &   -           &  -     &  -    & -         \\\hline               
 several time points    & +     &   +           &  +     &  -    & -         \\\hline                    
  variable design $^1$       & +     &   -           &  -     &  -    & -         \\\hline 
 non-constant rates     & -     &   +           &  +     &  -    & -         \\\hline 
            uridine bias& +     &   -           &  -     &  +    & +         \\\hline 
       RNA processing   &$\ast$ &   +           &  +     &  -    & -         \\\hline 
  language              & R     &MATLAB         &  R     &  R    & Java      \\\hline 
 \end{tabular}
\caption{Comparison of available software for parameter estimation in 
pulse-chase experiments. N: normal, NB: negative binomial, BIN: binomial.
$\ast$ - must be defined by a user. $^1$ - pulse, chase or combination thereof experiments.
}
\end{table}
We simulated count data for a pulse experiment 
in order to evaluate \verb|pulseR| performance.
The  detailed protocol of this procedure is described in the supplementary material.

%\section{Conclusion}
% The pulseR package provide an  interface for fitting models of
% RNA dynamics to the read count data. 
% It is freely available at 
% http://github.com/dieterich-lab/pulseR.