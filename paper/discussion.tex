\subsection{Comparison with existing approaches}
The published approaches are different in terms of 
data normalisation, statistical model and underlying mathematical model of the 
RNA metabolism. Here we analyse the following software:
DRiLL \citep{rabani2014high},
INSPEcT \citep{de2015inspect},
DTA \citep{schwalb2012measurement},
HALO \citep{friedel2010halo}.
\par 
In most cases, samples are normalised by utilising overdetermination of the system.
The normalisation coefficients are estimated via regression
(DTA, HALO) or during the MLE procedure together  
with other parameters (INSPEcT, DRiLL). 
Besides the normalisation during parameter fitting, 
pulseR allows to use spike-ins counts as an alternative.
\par
HALO and DTA estimate degradation rates from a ratio of labelled and
total RNA fractions without any assumptions on the statistical model.
In DRiLL, expression levels are fitted to the binomial distribution. 
However, the kinetic rates are estimated via
optimisation of residual sum of squares in both, DRiLL and 
INSPEcT,
which implies the normal distribution. 
In contrast,  pulseR assumes the NB distribution
for MLE of  all  parameters,
which allows to work directly on the count data.
\par
Experiments may vary in scheme and time points number,
and it is important how flexible to it a package is.
 DTA and HALO are designed to work only on a single time
point. DRiLL and INSPEcT can infer rates on the basis of several time points.
Moreover, the mentioned packages can work only with  the pulse-experiments.
pulseR package can handle different designs including chase- and combined 
experiments with various number of data points,
having formulas for mean read number estimation provided.	
\par 
The DRiLL and INSPEcT packages can model 
time-dependent rates out of the box. Additionally, they can perform testing
to select between constant rate and variable rate models.
Noteworthy, the DRiLL software is able to handle data about multiple mRNA isoforms.
\begin{table}
 \begin{tabular}{|l|c|c|c|c|c|}\hline
                        &pulseR &DRiLL          &INSPEcT&DTA    &HALO       \\\hline
 statistical model      & NB    &N, BIN         &N       & -    & -      \\\hline                         
 spike-ins              & +     &   -           &  -     &  -    & -         \\\hline               
 several time points    & +     &   +           &  +     &  -    & -         \\\hline                    
  variable design       & +     &   -           &  -     &  -    & -         \\\hline 
 non-constant rates     & -     &   +           &  +     &  -    & -         \\\hline 
            uridine bias& +     &   -           &  -     &  +    & +         \\\hline 
       RNA processing   &$\ast$ &   +           &  +     &  -    & -         \\\hline 
       gene isoforms    & -     &   +           &  -     &  -    & -         \\\hline 
  language              & R     &MATLAB         &  R     &  R    & Java      \\\hline 
 \end{tabular}
\caption{Comparison of available software for parameter estimation in 
pulse-chase experiments. N: normal, NB: negative binomial, BIN: binomial.
$\ast$ - must be defined by a user.
}
\end{table}

\subsection{Application}
We evaluated \verb|pulseR| performance using simulated data. 
The software is able to reproduce gene-specific parameters and 
sample normalisation factors.
For the detailed
description of the workflow please refer to the supplementary material.

%\section{Conclusion}
% The pulseR package provide an  interface for fitting models of
% RNA dynamics to the read count data. 
% It is freely available at 
% http://github.com/dieterich-lab/pulseR.